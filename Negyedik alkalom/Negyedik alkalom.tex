\documentclass{article}
\usepackage[english, magyar]{babel}
\usepackage{t1enc}
\usepackage{hulipsum}
\usepackage{algorithmicx}
\usepackage{amsthm}
\usepackage{float}
\usepackage{listings}
\usepackage{geometry}
\usepackage{algpseudocode}
\theoremstyle{plain}\newtheorem{tet}{Tétel}
\theoremstyle{definition}\newtheorem{defin}{Definíció}
\theoremstyle{plain}\newtheorem{lemma}{Lemma}
\theoremstyle{remark}\newtheorem{feladat}{Feladat}
\newtheorem*{megjegy}{Megjegyzés}
\newtheorem{lista}{Lista}
\newfloat{forráskód}{hbt}{lop}[section]
\geometry{margin=2cm}

\begin{document}
	\begin{titlepage}
		\title{\Huge{Bevezetés a \LaTeX-be}}
		\author{\Large{\textit{Negyedik alkalom}}}
		\date{}
		\maketitle
	\end{titlepage}

\section{feladat - Tételkörnyezetek}
\begin{tet}
	Tehát tétel
\end{tet}

\begin{tet}
	Heße mátrix	
\end{tet}

\begin{proof}
	...
\end{proof}

\begin{defin}
	Tehát definíció
\end{defin}

\begin{defin}
	Triviális
\end{defin}

\begin{lemma}
	Tehát Lemma
\end{lemma}

\begin{feladat}
	Feladat
\end{feladat}

\begin{megjegy}
	Megjegyzés
\end{megjegy}

\section{feladat - Verbatim (és saját float)}
\verb\LaTeX \LaTeX \LaTeX

\begin{verbatim}
	\begin{tet}
		Tétel...
	\end{tet}
\end{verbatim}

\begin{verbatim}
	\begin{lista}
		Lista...
	\end{lista}
\end{verbatim}

\listof{forráskód}{Programkódok listája}
\floatname{forráskód}{Forráskód}
\clearpage

\section{feladat - Programkód}
Former project from DBMS:
\begin{footnotesize}
	\lstinputlisting[language=Java, tabsize=4, numbers=left, stepnumber=4, frame=shadowbox]{App.java}
\end{footnotesize}
\clearpage

\section{feladat - Pszeudokód}
	\begin{lstlisting}[frame = single][language=python]
	def binary_search(arr, val, start, end):
	if start == end:
	if arr[start] > val:
	return start
	else:
	return start+1
	elif start > end:
	return start
	else: 
	mid = (start+end)/2
	if arr[mid] < val:
	return binary_search(arr, val, mid+1, end)
	elif arr[mid] > val:
	return binary_search(arr, val, start, mid-1)
	else: # arr[mid] = val
	return mid
	
	def insertion_sort(arr):
	for i in xrange(1, len(arr)):
	val = arr[i]
	j = binary_search(arr, val, 0, i-1)
	arr = arr[:j] + [val] + arr[j:i] + arr[i+1:]
	return arr
\end{lstlisting}

\begin{algorithmic}
	\State $i \gets 10$
	\If{$i\geq 5$} 
	\State $i \gets i-1$
	\Else
	\If{$i\leq 3$}
	\State $i \gets i+2$
	\EndIf
	\EndIf 
\end{algorithmic}

\end{document}
