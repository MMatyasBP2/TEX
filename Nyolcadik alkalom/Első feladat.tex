\documentclass[aspectratio=169, bigger, xcolor={table}]{beamer}
\usepackage[magyar]{babel}
\usepackage{t1enc}
\usepackage{lipsum}
\usepackage{hulipsum}
\usepackage{mathtools}
\usepackage{colortbl}
\usepackage{amsthm}

\usetheme{CambridgeUS}
\usecolortheme{spruce}

\author{Martinák Mátyás}
\title{Feladatok}
\subtitle{A feladatok megoldása}
\institute{Miskolci Egyetem}
\date{\today}


\begin{document}
	\maketitle
	
	\begin{frame}{A keret címe}{És az alcím}
		Ez itt ennek a keret-nek a tartalma.
		
		A feladat szerint írni kell néhány sort.
	\end{frame}
	
	\begin{frame}[allowframebreaks]{Második keret}{2. alcím}
		\hulipsum
	\end{frame}
	
	\begin{frame}{Harmadik keret}{Harmadik keret alcíme}
		Ez itt ennek a keret-nek a tartalma.
		
		A feladat szerint írni kell néhány sort.
	\end{frame}
	
	\begin{frame}{Negyedik keret}{Negyedik keret alcíme}
		Ez itt ennek a keret-nek a tartalma.
		
		A feladat szerint írni kell néhány sort.
	\end{frame}
	
	\begin{frame}[fragile]{Verbatimos keret}{Verbatim alcím}
		
		\begin{verbatim}
			\fjkd
			slskjf
			\ksdjfk
			\ksdj
		\end{verbatim}
		
	\end{frame}
	
\end{document}