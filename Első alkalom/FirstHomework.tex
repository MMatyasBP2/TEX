\documentclass{book}
\usepackage[english, magyar]{babel}
\usepackage{t1enc}
\usepackage{xcolor}
\usepackage{graphics}
\usepackage{lipsum}
\usepackage{hulipsum}
\usepackage{blindtext}
\frenchspacing
\begin{document}
	\begin{titlepage}
		\begin{center}
			\title{MainTitle}\textbf{\huge{\textsc{Bevezetés a \LaTeX-be\\
			\textbf{I. alkalom}}}}
		\end{center}
	\footnotetext{\author{Author}\textit{Írta: Martinák Mátyás}}
	\end{titlepage}
\title{Title}\textbf{\huge{\textit{Bevezetés a \LaTeX-be első óra}}}
\author{Mátyás Martinák}
\section{Első feladat (Létrehozás, lementés)}
\section{Második feladat (Betűtípusok)}
	\texttt{Typewriter}\\
	\emph{\textbf{Kopasz}}\\
	\emph{\textsc{Kiskapi}}\\
	\emph{\textit{Italic}}\\
	\textsl{Slanted}\\
	\textsf{Sant serif}\\\\
	A \textbf{Kopasz} betűtípust bedönti, az \textit{Italic} betűtípust és a \textsc{Kiskapitálist} nem.
\section{Harmadik feladat (Szöveg dekoráció)}
	A színekhez és a tükrözéshez az xcolor és a graphics csomagokra van szükségünk.\\
	A tükrözött és keretezett szó: \framebox{\reflectbox{\LaTeX}}\\
	Tükrözött és vízszintesen megnyújtott szó: \scalebox{2.0}{\scalebox{-1}[1]{\hulipsum[1]}}\\
	90°-kal elforgatott szó: \rotatebox{90}{\LaTeX}\\
	270°-kal elforgatott szó: \rotatebox{270}{\LaTeX}\\
	Invertált szöveg:\\
	\colorbox{black}{\textcolor{white}{\LaTeX}}\\
	Piros szöveg, piros kerettel, háttér nélkül:\\
	\fcolorbox{red}{white}{\textcolor{red}{\LaTeX}}\\
\section{Negyedik feladat (Zagyva szöveg/lorem ipsum)}
	\hulipsum[1]\\\\
	\textbf{És most latinul is:}\\
	\begin{flushright}
		\begin{otherlanguage}{latin}
			\lipsum[1]
		\end{otherlanguage}
	\end{flushright}
	\linespread{2}
\section{Ötödik feladat (babel és központozás)}
	A frenchspacing-el a térközök változtak. A latin nyelv benne van az alap nyelvi csomagban.\\\\
	Mai nap: "\today"\\
	\begin{otherlanguage}{english}
		Today: "\today"
	\end{otherlanguage}\\\\
	Láthatjuk, hogy az angol más dátumformázást használ, mint a magyar.
\section{Hatodik feladat (Cím)}
	Az article és a book documentclassok közti különbség, az, hogy a book documentclass kvázi könyvként kezeli a dokumentumot, így oldalanként máshova rendezi a beírt szöveget.
\end{document}