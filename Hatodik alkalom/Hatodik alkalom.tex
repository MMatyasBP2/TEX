\documentclass{article}
\usepackage[magyar]{babel}
\usepackage{t1enc}
\usepackage{lipsum}
\usepackage{hulipsum}
\usepackage{amsmath}
\usepackage{amsfonts}
\usepackage{amssymb}
\usepackage{mathtools}

\begin{document}
	\begin{titlepage}
		\title{\Huge{Bevezetés a \LaTeX-be}}
		\author{\Large{\textit{Matematikai szövegek}\\
				\textbf{by Martinák Mátyás}}}
		\date{}
		\maketitle
	\end{titlepage}
	\textbf{Bevezető}\\
	
	Az $\frac{1}{n^2} $ sor összege:
	
	\[\sum_{i=1}^\infty\frac{1}{n^2}=\frac{\pi^2}{6}\]
	
	Az $n!$ ($n$ faktoriális) a számok szorzata 1-től n-ig, azaz
	\begin{equation}
		n! := \prod_{i=1}^n k = 1 \cdot 2 \cdot ...\cdot n\text{.}
	\end{equation}
	
	Konvekció szerint $0! = 1$.
	
	Legyen $0 \leq k \leq n$. A binomiális együttható
	\[\binom{n}{k} = \frac{n!}{k! \cdot (n-k)!}\text{,}\]
	
	ahol a faktoriálist (1) szerint definiáljuk.
	
	Az előjel- azaz szignum függvényt a következőképpen definiáljuk:
	\[\operatorname{sgn}(x) := \begin{cases}
		1, & \text{ha } x > 0, \\
		0, & \text{ha } x = 0, \\
		-1, & \text{ha } x < 0.
	\end{cases}\]
	
	\textbf{Determináns}\\
	
	Legyen
	\[[n] := \lbrace1, 2, ..., n\rbrace\]
	
	a természetes számok halmaza 1-től n-ig.
	
	Egy $n$-edrendű $\text{permutáció }\sigma$ egy bijekció $[n]$-ből $[n]$-be. Az $n$-edrendű permutációk halmazát, az ún. szimmetrikus csoportot, $S_n$-nel jelöljük.
	
	Egy $\sigma \in S_n$ permutációban inverziónak nevezünk egy $(i, j)$ párt, ha $i < j$
	de $\sigma i > \sigma j$.
	
	Egy $\sigma \in S_n$ permutáció paritásának az inverziók számát nevezzük:
	\[I(\sigma) := \vert\lbrace(i, j) \vert i, j \in [n], i < j, \sigma_i > \sigma_j \rbrace\vert.\]
	
	Legyen $A \in \mathbb{R}^{n \times n} \text{, egy }n \times n$-es (négyzetes) valós mátrix:
	\[ A = \left( \begin{matrix}
		a_11 & a_12 & \cdots & a_1n \\
		a_21 & a_22 & \cdots & a_2n \\
		\vdots & \vdots & \ddots & \vdots \\
		a_n1 & a_n2 & \cdots & a_nn \\
	\end{matrix} \right) \]
	
	Az $A$ mátrix determinánsát a következőképpen definiáljuk:
	\begin{equation}
		\operatorname{det}(A) =  \begin{vmatrix}
			a_11 & a_12 & \cdots & a_1n \\
			a_21 & a_22 & \cdots & a_2n \\
			\vdots & \vdots & \ddots & \vdots \\
			a_n1 & a_n2 & \cdots & a_nn \\
		\end{vmatrix} := \sum_{\sigma \in S_n} (-1)^{I(\sigma)} \prod_{i=1}^n a_{i\sigma_i}
	\end{equation}
	
	\textbf{Logikai azonosság}\\
	
	\noindent Tekintsük az $L = \lbrace0, 1\rbrace$ halmazt, és rajta a következő, igazságtáblával definiált műveleteket:
	
	\[\begin{array}{c || c}
		x & \bar{x} \\ \hline
		0 & 1 \\
		1 & 0 \\
	\end{array}
	\quad
	\begin{array}{c c || c | c | c}
		x & y & x \vee y & x \wedge y & x \rightarrow y \\ \hline
		0 & 0 & 0 & 0 & 1 \\
		0 & 1 & 1 & 0 & 1 \\
		1 & 0 & 1 & 0 & 0 \\
		1 & 1 & 1 & 1 & 1 \\
	\end{array}
	\]
	
	Legyenek $a, b, c, d \in L$. Belátjuk a következő azonosságot:
	
	\begin{equation}
		(a \wedge b \wedge c) \rightarrow d = a \rightarrow ( b \rightarrow( c \rightarrow d))\text{.}
	\end{equation}
	
	A következő azonosságokat bizonyítás nélkül használjuk:
	
	\begin{subequations}
		\begin{equation}
			x \rightarrow y = \bar{x} \vee y
		\end{equation}
		\begin{equation}
			\overline{x \vee y} = \bar{x} \wedge \bar{y}
			\quad
			\overline{x \wedge y} = \bar{x} \vee \bar{y}
		\end{equation}
	\end{subequations}
	
	A (3) bal oldala, (4) felhasználásával
	
	\begin{equation}
		(a \wedge b \wedge c) \rightarrow d = \overline{a \wedge b \wedge c} \vee d = (\bar{a} \vee \bar{b} \vee \bar{c}) \vee d \text{.}
	\end{equation}
	
	A (3) jobb oldala, (4a) ismételt felhasználásával
	
\end{document}