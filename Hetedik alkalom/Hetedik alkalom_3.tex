\documentclass{article}
\usepackage[magyar]{babel}
\usepackage{t1enc}
\usepackage{lipsum}
\usepackage{hulipsum}
\usepackage{amsmath}
\usepackage{amsfonts}
\usepackage{amssymb}
\usepackage{mathtools}

\newenvironment{vonalzott}%
{\vspace{1ex}\hrule\vspace{1ex}}%
{\vspace{1ex}\hrule\vspace{1ex}}

\newenvironment{vonalzott*}[1][Kulcsgondolatok]{\begin{center}\begin{minipage}{0,8\textwidth}%
			\vspace{1ex}\hrule\vspace{1ex}\begin{center}\Large\textbf{#1}\end{center}}{\vspace{1ex}\hrule\vspace{1ex}%
\end{minipage}\end{center}}


\begin{document}
	
	\begin{vonalzott*}
		\newcommand{\kgitem}{\par\makebox[1 cm]{\stepcounter{szamlalo} \theszamlalo}}
		\newcounter{szamlalo}
		\kgitem
		\section{Elso}
		\kgitem
		\hulipsum[4]
		\section{Masodik}
		\kgitem
		\hulipsum[2]
		\kgitem
		\kgitem kdlsjf
		\kgitem lkdsa
		A számláló értéke: \theszamlalo
	\end{vonalzott*}
	
\end{document}