\documentclass[twocolumn]{book}
\usepackage[english, magyar]{babel}
\usepackage{t1enc}
\usepackage{hulipsum}
\usepackage[inner=3cm,outer=5cm]{geometry}

\geometry{bindingoffset=0cm}
\geometry{marginparwidth=3cm, marginparsep=0.5cm}
\setcounter{secnumdepth}{5}
\frenchspacing

\begin{document}
	\begin{titlepage}
		\title{\textsc{\huge{Bevezetés a \LaTeX\ \\ szövegszerkesztőbe}}}
		\author{\textit{\Large{Második alkalom}}}
		\maketitle
		\pagenumbering{roman}
	\end{titlepage}

\renewcommand{\thefootnote}{\fnsymbol{footnote}}

\setcounter{tocdepth}{5}
\tableofcontents
\clearpage
\pagenumbering{arabic}
\section{feladat}\footnote{Section footnote}
A titlepage elkészítése a maketitle paranccsal. A subsection-ben a számozás a pont után folytatódik.
\subsection{Első alszekció}
\hulipsum[2]
\clearpage
\subsection{Második alszekció}
\hulipsum[3]
\section{feladat - Hosszabb szekció elkészítése}
\part{Első szint}
\section{Második szint}
\subsection{Harmadik szint}
\subsubsection{Negyedik szint}
\paragraph{Ötödik szint}
\subparagraph{Hatodik szint}

\appendix

\section{section}
\begin{quote}
	\hulipsum[1-2]
\end{quote}
\subsection{subsection}
\section{section}
\begin{quotation}
	\hulipsum[1-2]
\end{quotation}

\clearpage
\marginpar{Margó szöveg}
\subsection{József Attila - Tiszta szívvel}
\begin{verse}
	\begin{center}
		Nincsen apám, se anyám,\\
		se istenem, se hazám,\\
		se bölcsőm, se szemfedőm,\\
		se csókom, se szeretőm.\\
		
		Harmadnapja nem eszek,\\
		se sokat, se keveset.\\
		Húsz esztendőm hatalom,\\
		húsz esztendőm eladom.\\
		
		Hogyha nem kell senkinek,\\
		hát az ördög veszi meg.\\
		Tiszta szívvel betörök,\\
		ha kell, embert is ölök.\\
		
		Elfognak és felkötnek,\\
		áldott földdel elfödnek\\
		s halált hozó fű terem\\
		gyönyörűszép szívemen.\\
	\end{center}
\end{verse}

	
\end{document}